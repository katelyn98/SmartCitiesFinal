\documentclass[12pt]{article}
\usepackage[margin=1in]{geometry}
\usepackage{setspace}
\usepackage{indentfirst}

\doublespacing
\title{
    How Attributes of a Smart City are Scaffolding Tomorrow's Culture
}
\author{Katelyn C Morrison}

\begin{document}
\maketitle

\section*{Introduction}
Will privacy be invaded? Will ethics be shattered? Who has the ultimate control? How
is this technology positively contributing to my existence in society?
These are all questions that are running through the mind of many citizens today. 
Many citizens are not exposed to the interworkings behind the functioning of their city.
However, many passionate scientists are collaborating to allow the city to have power...to allow the city to predict, analyze, and adapt.
Smart cities are unqieu to their distinct offerings, but their structure remains homogenous. 
And during a pandemic, vital components of a smart city like infrastructure, transportation, 
health care, governance, education, and technology are now being tested to their potentials \cite{DefiningSmartCities}.
Novel research methods using sensors, deep learning, and statistical learning are continuing to push the status quo. 
For example, models can now predict
when toilet paper will run out and at which stores \cite{GotTP}. Other models can analyze and determine
social distancing standards during pandemics \cite{das_james_2020}. Through geotags and informatics,
models can present data trends on how well neighborhoods are aligning to social distancing policies \cite{gazette_2020}.
Where does it cross the line? When is it categorized as invading privacy? Does this present a
stronger case for the need of open source datasets populated through crowdsourcing techniques? 
Through the analysis of data security, 
sensor technology, and the power of prediction, it will be made clear how tomorrow's
cultural norms are being scaffolded by today's smart city and machine learning researcher.
\subsection*{Data Trends in Transportation \& Mobility}
Historically, it can be understood that humans are nomadic. They continue to have an urge or desire
to relocate, travel, and change scenery. These desires led to iterations of bikes, cars, and planes
with the most successful prototypes redefining the culture of the future. This also applies to the design 
of transportation systems in urban centers to handle large masses of people. 

-talk about the social distancing trends relating to Mobility and things observed

\subsection*{Open Source Data: Pros \& Cons}


\subsection*{Neighborhood Anaylsis \& Urbanization}

\subsection*{Conclusion}

\bibliographystyle{acm}
\bibliography{references}


\end{document}