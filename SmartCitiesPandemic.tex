\documentclass[12pt]{article}
\usepackage[margin=1in]{geometry}
\usepackage{setspace}
\usepackage{indentfirst}

\doublespacing
\title{
    How Attributes of a Smart City are Scaffolding Tomorrow's Culture
}
\author{Katelyn C Morrison}

\begin{document}
\maketitle

\section*{Introduction}
Will privacy be invaded? Will ethics be shattered? Who has the ultimate control? How
is this technology positively contributing to my existence in society?
These are all questions that are running through the mind of many citizens today. 
Many citizens are not exposed to the innerworkings behind the functioning of their city.
However, many passionate scientists are collaborating to allow the city to have power...to allow the city to predict, analyze, and adapt.
Smart cities are unqiue to their distinct offerings, but their structure remains homogenous. 
And during a pandemic, vital components of a smart city like infrastructure, transportation, 
health care, governance, education, and technology are now being put to the test \cite{DefiningSmartCities}.
Novel research methods using sensors, deep learning algorithms, and open source datasets are continuing to push the status quo. 
For example, models can now predict
when toilet paper will run out and at which stores \cite{GotTP}. Other models can analyze and determine
social distancing standards during pandemics \cite{das_james_2020}. Through geotags and informatics,
models can present data trends on how well neighborhoods are aligning to social distancing policies \cite{gazette_2020}.
Where does it cross the line? When is it categorized as invading privacy? Or does this present a
stronger case for the need of open source datasets populated through crowdsourcing techniques? 
Through the analysis of socio-technical systems, location data analysis, and sensor technology, 
it will be made clear how tomorrow's
cultural norms are being scaffolded by today's researchers and to what extent individuals 
privacy is invaded.
\subsection*{Socio-Technical Systems and Mobility}
A smart city's socio-technical system is crafted in a way such that citizens, software, hardware, data, and policies are all connected \cite{rangwala_2018}.
In light of the Coronavirus Pandemic, cities are discovering and taking advantage of their
socio-technical systems in hopes to mitigate and avoid the spread of the virus. "Mobility systems are socio-technical systems...", Debra Lam points out in an article about
a project at Georgia Tech's Socially Aware Mobility Lab (SAM) \cite{levine_2020}. Not only is it important to make transportation easily accessible and sustainable,
but it is now vital to create a transportation infrastructure that promotes social distancing and well-being. Walker brings up a good point that, "...
if we all drive cars out of a feeling of personal safety, we'll quickly restore the congestion that strangle our cities" \cite{walker_2020}. How are local governments
harnessing their city's soci-technical system to face the challenge of sucha  large decline in their ridership? From my extensive research on the topic, it remains unclear.
Several transit agencies have created cleaning crews to perform deep cleaning and some cities have even given the drivers permission to skip stops if their vehicle is going 
to exceed social distancing standards \cite{hawkins_2020}. This might open up the doors for Uber and Lyft to pave the way in offering on-demand mobility as a public transit 
option. This disruption is leaving people questioning how 
the transportation infrastructure and systems can change and adapt to create the transportation system of the future. 

Another component of socio-technical systems, data 
and policies, brings light on the power that technical researchers have over cities and its citizens. Several maps and models have been created and 
released to present mobility trends since the spread of the virus accelerated. 


For example, The Newcastle University Urban
Observatory developed a model backed by convolutional neural networks to measure the distance between pedestrians at a once 
busy intersection. 


\subsection*{'Your Location is Showing'}


\subsection*{Neighborhood Anaylsis \& Urbanization}

\subsection*{Conclusion}

\bibliographystyle{acm}
\bibliography{ref}


\end{document}